\documentclass[12pt, a4paper]{article}
\usepackage{fontspec} % loaded by polyglossia, but included here for transparency 
\defaultfontfeatures{Mapping=tex-text}
\usepackage{polyglossia}
\usepackage{microtype}
\usepackage{hyperref}
\usepackage{xifthen}
\hypersetup{
	colorlinks=true, linkcolor = {black}, urlcolor = {blue}, breaklinks=true
}
\setlength{\parskip}{.5em}
\setmainlanguage{russian} 
\setotherlanguage{english}
\usepackage{enumitem}

% XeLaTeX can use any font installed in your system fonts folder
% Linux Libertine in the next line can be replaced with any 
% OpenType or TrueType font that supports the Cyrillic script.

\newcounter{problem}
\newcommand{\problem}[1]{\refstepcounter{problem}{\bf Задача \theproblem.} \ifthenelse{\isempty{#1}{}}{}{\label{#1}}}

\newfontfamily\russianfont{Times New Roman}
\newfontfamily\englishfont{Times New Roman}
\setmonofont{Courier New}
\newfontfamily{\cyrillicfonttt}{Courier New}

\title{Задания по курсу Python\\Задание 3}
\author{Д.В. Иртегов}
\date{\today}

\begin{document}
\pagestyle{empty}
\maketitle
{\small Задачи необходимо сдать до 14 апреля.  Решения необходимо сдавать путем отправки pull request в каталог problems-3 репозитория \\
\url{https://github.com/dmitry-irtegov/NSUPython2018}. 
\\Датой сдачи задания считается дата отправки первого pull request.  Если запрос не принят из-за моих замечаний, у вас есть неделя на их исправление. 

Если запрос принят, задание считается засчитанным.  Если запрос не принят, в комментарии вы можете узнать, почему.

В одном запросе следует отправлять не более одного решения. Если решение состоит из нескольких файлов, в запрос должны быть включены они все.  Все запросы одного студента должны отправляться в каталог с именем, соответствующим его учетной записи.  Например, для задачи 3 из группы задач 2, сдаваемой студентом v-pupkin, рекомендуемое имя файла \verb|problems-3/v-pupkin/task3.py|. }

\problem{average} Напишите скрипт, который считывает файл данных.  Файл имеет формат, аналогичный \url{http://parallels.nsu.ru/~fat/Python/log-2}. Можете предполагать, что файл содержит только символы ASCII.  Обратите внимание на размер файла!  Вам необходимо:
\begin{itemize}[noitemsep,topsep=0em]
\item Выбрать все строчки, начинающиеся со слова open.
\item В каждой такой строчке, выбрать время (число, размещенное перед словом usec).  Эти времена и есть значения, которые вам необходимо обработать.
\item Отбросить первое значение (в примере это 15685 usec).
\item Подсчитать среднее всех остальных значений и их среднеквадратичное отклонение.
\item Вывести вычисленные значения.
\end{itemize}
Задание необходимо выполнить без хранения всех значений одновременно.
\pagebreak

\problem{} Для того же файла данных, что и в задании \ref{average}, вместо среднеквадратичного значения подсчитайте верхний дециль, то есть такое число, что 90\% 
всех значений не превосходят этого числа.  Определить это значение без хранения хотя бы части данных, насколько я знаю, невозможно.  Храните только минимально необходимые данные и используйте для их хранения оптимальную по асимптотической производительности структуру.

\problem{} Реализуйте ленивое буферизованное чтение из файла.  Реализуйте генератор или итерабельный объект, которому в качестве параметра конструктора передается открытый файл.  Далее этот объект должен последовательно возвратить все байты или символы (в зависимости от того, в каком режиме был открыт файл) этого файла, включая переводы строки.  При этом, объект должен читать данные из файла блоками по 512 байт или символов.

\problem{} В задаче <<Вектор>> из группы заданий 2, реализуйте полиморфный конструктор. Если в качестве параметра передана коллекция, итерабельный объект или генератор с элементами, приводимыми к числовому типу, то должен создаваться вектор той же размерности, что и мощность коллекциии.  При этом, значения координат вектора должны быть равны значениям элементов коллекции/итерабельного объекта.  Если передано $N$ скалярных параметров, приводимых к числовому типу, должен создаваться вектор размерности $N$ с соответствующими значениями.  Во всех остальных случаях должно выкидываться исключение \verb|TypeError|.

Во всех случаях, когда вектор создается, все элементы вектора должны быть приведены к одному типу, соответствующему самому <<сложному>> из переданных в качестве параметра.  <<Сложность>> числового типа определяется списком \verb|int, float, complex|, где \verb|int| -- самый <<простой>>, а \verb|complex| -- самый <<сложный>>

\problem{} В задаче <<Вектор>> из группы заданий 2, реализуйте полиморфную операцию умножения.  Если этой операции передан вектор той же размерности, должно производиться скалярное умножение.  Если передано число или значение, приводимое к числу, должно проводиться умножение на скаляр.  Во всех остальных случаях, должно выбрасываться исключение \\ \verb|TypeError|.

\problem{} Реализуйте класс Vector3D, который является наследником вашего класса Vector.  Этот класс должен допускать только трехмерные вектора, и для него должна быть дополнительно опредена операция векторного произведения.

\end{document}
